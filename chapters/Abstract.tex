\label{Abstract}
\addcontentsline{toc}{chapter}{\numberline{}Abstract}
\begin{abstract}

Nowadays, many efforts are put into the development of intelligent cars as well as small autonomous mobile vehicles for various purposes (e.g. military, agriculture, household robotics).Purpose of the document is to describe the methods and the different development stages of  the project "Escaper bot",an autonomous robotic vehicle  that keeps a desired distance (escapes) from objects placed in front of it. 
\\
Furthermore,details about key components and circuit implementation to achieve this goal are described. 
\end{abstract}

\section{Methodology}
Before going deepen the project development, a plan of work and goals set have to be explained . Then after setting the main goal, in order to achieve it,the project has been sliced into different sub parts:

\begin{enumerate}
  \item Design H-bridge and motor control (run the DC motors, H-bridge, security, quality)
  \item Design the sensor part (IR sensor interfacing, ADC interfacing and controlling with FPGA)
  \item A proper robot chassis
  \item Control system with the FPGA
\end{enumerate}

The project has been structured in the following way:a list of necessary components has been compiled,each of them has been tested and a measurement of their ratings has been made with the available laboratory equipment.After the design of the different circuit composing the robot and performance test,the PCB boards have been printed and and constructed,trying to minimize the different sizes.Furthermore,some boards (A/D converter board,H-Bridges) have been tested singularly with the FPGA,with an early implementation of the code to make sure that the behaviour have been corresponding to the set expectation (i.e. the H-Bridge has been tested with the PWM generator implemented on the FPGA,in order to highlight and consequentially improve hypothetical weak points of the hardware configuration).

